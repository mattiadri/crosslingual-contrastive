\section{Results}

\subsection{Multilingual Image--Text Retrieval Performance}

We first evaluate our approach on the task of multilingual image--text retrieval, which directly measures how well textual representations in different languages align with a shared visual embedding space. We report results for the text-to-image direction, using Recall@1, Recall@5, Recall@10, and Mean Reciprocal Rank (MRR). All metrics are computed per language and macro-averaged across languages.

\subsubsection{Overall Performance}
\begin{table}[t]
\centering
\small
\begin{tabular}{lccc}
\toprule
 & Identity (Before) & After (Linear) & After (MLP) \\
\midrule
R@1  & $0.4735 \pm 0.0077$ & $0.4938 \pm 0.0053$ & $0.5049 \pm 0.0070$ \\
R@5  & $0.7284 \pm 0.0052$ & $0.7467 \pm 0.0044$ & $0.7505 \pm 0.0054$ \\
R@10  & $0.8091 \pm 0.0067$ & $0.8239 \pm 0.0042$ & $0.8264 \pm 0.0066$ \\
MRR  & $0.5898 \pm 0.0043$ & $0.6079 \pm 0.0027$ & $0.6159 \pm 0.0036$ \\
\bottomrule
\end{tabular}
\caption{Multilingual text-to-image retrieval on holdout folds.
Mean $\pm$ std over 5-fold cross-validation, macro-averaged across 9 languages.}
\label{tab:cv_retrieval_macro}
\end{table}

\vspace{0.5em}

\begin{table}[t]
\centering
\resizebox{\linewidth}{!}{%
\begin{tabular}{lcccccccccc}
\toprule
 & \multicolumn{5}{c}{R@1} & \multicolumn{5}{c}{MRR} \\
\cmidrule(lr){2-6} \cmidrule(lr){7-11}
Lang & Before & Linear & MLP & $\Delta$Lin & $\Delta$MLP & Before & Linear & MLP & $\Delta$Lin & $\Delta$MLP \\
\midrule
ar & $0.3335$ & $0.3584$ & $0.3670$ & +0.0249 & +0.0336   & $0.4474$ & $0.4715$ & $0.4781$ & +0.0241 & +0.0306 \\
de & $0.5043$ & $0.5231$ & $0.5296$ & +0.0188 & +0.0253   & $0.6174$ & $0.6336$ & $0.6410$ & +0.0162 & +0.0236 \\
en & $0.6019$ & $0.6109$ & $0.6246$ & +0.0090 & +0.0228   & $0.7101$ & $0.7183$ & $0.7286$ & +0.0082 & +0.0184 \\
es & $0.5159$ & $0.5390$ & $0.5546$ & +0.0231 & +0.0387   & $0.6307$ & $0.6488$ & $0.6606$ & +0.0181 & +0.0299 \\
fr & $0.5105$ & $0.5242$ & $0.5473$ & +0.0137 & +0.0368   & $0.6260$ & $0.6391$ & $0.6529$ & +0.0131 & +0.0269 \\
it & $0.4993$ & $0.5191$ & $0.5289$ & +0.0199 & +0.0296   & $0.6170$ & $0.6341$ & $0.6409$ & +0.0170 & +0.0239 \\
ja & $0.3562$ & $0.3837$ & $0.3923$ & +0.0275 & +0.0361   & $0.4786$ & $0.5069$ & $0.5111$ & +0.0283 & +0.0325 \\
pt & $0.5228$ & $0.5437$ & $0.5560$ & +0.0210 & +0.0332   & $0.6375$ & $0.6534$ & $0.6621$ & +0.0159 & +0.0247 \\
zh & $0.4169$ & $0.4422$ & $0.4433$ & +0.0253 & +0.0264   & $0.5439$ & $0.5652$ & $0.5673$ & +0.0213 & +0.0235 \\
\bottomrule
\end{tabular}
}
\caption{Per-language retrieval performance on holdout folds.
Deltas are computed w.r.t.\ the identity baseline.}
\label{tab:cv_retrieval_perlang}
\end{table}


Unless otherwise specified, all retrieval results are reported as mean $\pm$ standard deviation over 5-fold cross-validation, macro-averaged across the 9 languages (ar, de, en, es, fr, it, ja, pt, zh).

We observe consistent improvements after alignment across all metrics. In particular, Linear improves macro R@1 by 0.0204 and macro MRR by 0.0180, while MLP yields larger gains (0.0314 R@1, 0.0260 MRR), and the effect is stable across folds, as indicated by the low standard deviation in Table~\ref{tab:cv_retrieval_macro}.

Since we adapt only a lightweight projection head on top of frozen encoders, absolute improvements are bounded; nonetheless, gains are consistent across folds and languages, indicating a robust alignment effect.

Interestingly, the MLP head yields the strongest gains in retrieval performance, whereas the linear residual head produces the most consistent improvements in cross-lingual representational similarity (RSA), indicating a trade-off between instance-level retrieval accuracy and global structural alignment.

\subsubsection{Per-Language Analysis}

Table~\ref{tab:cv_retrieval_perlang} reports Recall@1 and MRR (equivalent to mAP in our single-positive retrieval setup). for each individual language. Performance improvements are observed across all languages, including those with different scripts and morphological properties.

Languages with weaker initial alignment, such as Arabic, Japanese, and Chinese, benefit particularly from image-pivot training, showing larger relative gains in both Recall@K and MRR. High-resource languages such as English, German, and French also improve, albeit with smaller absolute gains, reflecting their stronger initial alignment inherited from pretrained representations.

This pattern suggests that image-pivot supervision is especially effective at reducing cross-lingual disparities, narrowing the performance gap between high-resource and lower-resource languages.

\subsubsection{Effect of Balanced Multilingual Batching}

We use language-balanced batching (equal number of samples per language within each batch) to prevent high-resource languages from dominating the contrastive objective. We found this choice important for stable multilingual alignment in practice.

\subsubsection{Summary}

Overall, these results demonstrate that image-pivoted contrastive learning is sufficient to induce strong multilingual alignment, yielding consistent retrieval improvements across a diverse set of languages without relying on parallel text or translation-based supervision.